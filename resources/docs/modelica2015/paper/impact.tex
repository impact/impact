%sample file for Modelica 2015 Conference paper

\documentclass[11pt,a4paper,twocolumn]{article}
\usepackage{graphicx}
% uncomment according to your operating system:
% ------------------------------------------------
%\usepackage[latin1]{inputenc}    %% european characters can be used (Windows, old Linux)
\usepackage[utf8]{inputenc}     %% european characters can be used (Linux)
%\usepackage[applemac]{inputenc} %% european characters can be used (Mac OS)
% ------------------------------------------------
\usepackage[T1]{fontenc}     %% get hyphenation and accented letters right
\usepackage{mathptmx}        %% use fitting times fonts also in formulas
\usepackage{amsmath,amssymb} %% Nice maths
\usepackage[round]{natbib}   %% author-year style referencing
\usepackage{doi}             %% Create cor­rect hy­per­links for DOI num­bers
\usepackage{booktabs}        %% Nice tables
\usepackage{hyperref}
\usepackage{color}
\usepackage[labelfont=bf, labelsep=period, font=small]{caption}  %% Get bold Figure/Table caption
               %% Set separator in figures to '.', set fontsize to small
\usepackage{authblk}         %% Prepare author and affiliation blocks
\usepackage{courier}         %% For proper courier fonts in texttt
\usepackage{listings}        %% For code sections
%\usepackage[bw]{dtsyntax}    %% For Modelica code

% do not change these lines:
\pagestyle{empty}                %% no page numbers!
\usepackage{geometry}            %% please don't change geometry settings!
\geometry{left=20mm, right=20mm, top=25mm, bottom=25mm, noheadfoot}

\hypersetup{
  pdftitle = {Where impact got going},
  pdfauthor = {Michael Tiller and Dietmar Winkler},
  pdfsubject = {11th International Modelica Conference 2015},
  pdfkeywords = {Modelica, package manager, GitHub, dependency management, golang},
  hidelinks,
  pdfpagelayout=SinglePage}

\renewcommand{\normalsize}{\fontsize{10.5pt}{12.3pt}\selectfont}
\renewcommand{\small}{\fontsize{9.5pt}{11.1pt}\selectfont}
\renewcommand{\footnotesize}{\fontsize{8.5pt}{9.9pt}\selectfont}

%% Modelica code configuration
% \lstset{language = Modelica,
%        basicstyle=\fontsize{9pt}{10.5pt}\selectfont,
%        backgroundcolor = \color{white}}

% usefull commands
\newcommand{\myr}{\textsuperscript{\textregistered}}
\newcommand{\ud}{\mathrm{d}}
\newcommand{\matx}[1]{\mathbf{#1}}
\newcommand{\impact}{\texttt{impact}} % impact is going to get used quite a lot :)
\newcommand{\code}[1]{\texttt{#1}} % make quoting code text a bit simpler


% begin the document
\begin{document}
\thispagestyle{empty}

\title{\textbf{Where \impact\ got \emph{go}ing}}
\renewcommand\Authfont{\large}        %% Set author font
\renewcommand\Affilfont{\normalsize}       %% Set affiliation font
\renewcommand\Authsep{\quad}                     %% Set text between authors names
\renewcommand\Authand{\quad}                     %% Set text between authors names
\renewcommand\Authands{\quad}                    %% Set text between authors names
\author[1]{Michal Tiller}
\author[2]{Dietmar Winkler}
\author[3]{Christoph H\"oger}
\affil[1]{\href{http://xogeny.com}{Xogeny Inc.}, USA, {\small \href{mailto:michael.tiller@xogeny.com}{\nolinkurl{michael.tiller@xogeny.com}}}}
\affil[2]{\href{http://www.hit.no}{Telemark University College}, Norway, {\small\href{mailto:dietmar.winkler@hit.no}{\nolinkurl{dietmar.winkler@hit.no}}}}
\affil[3]{\href{http://www.uebb.tu-berlin.de}{Technische Universit\"at Berlin}, Germany, {\small\href{mailto:christoph.hoeger@tu-berlin.de}{\nolinkurl{christoph.hoeger@tu-berlin.de}}}}
% \title{\textbf{Int. Modelica Conf. 2015 Paper Title}}
% \author{{\large
% Author Name$^1$ \quad Author Name$^1$ \quad Author Name$^2$\vspace{2mm}\\
%   {}$^1$Department, University, Country, \textsf{\{name1,name2\}@university.org}\\
%   {}$^2$Company, Contry, \textsf{name3@company}}


\date{} % <--- leave date empty
\maketitle\thispagestyle{empty} %% <-- you need this for the first page
\abstract{

  This paper discussed the \code{impact} package manager.  The primary
  goal of this project is to support the development of a healthy
  eco-system around Modelica.  For many other languages, the existance
  of an easy to use package manager has made it easier for people to
  explore and adopt those languages.  We seek to bring that same kind
  of capability to the Modelica community by incorporating useful
  features from other package managers like \code{bower}, \code{npm},
  {\it etc.}

  This paper is an update on the status of the \code{impact} package
  manager which was discussed previously in \cite{impact1}.  This
  latest version of \code{impact} involves a complete rewrite that
  incorporates a more advanced dependency resolution algorithm.  That
  dependency resolution will be discussed in depth along with many of
  the subtle issues that arose during the development of this latest
  version of impact.  Along with a superior dependency resolution
  scheme, the new version of \code{impact} is much easier to install
  and use.  Futhermore, it includes many useful new features as well.

}

\noindent\emph{Keywords: Modelica, package management, GitHub, dependency resolution, golang}

\section{Introduction}

\subsection{Motivation}

The motivation behind the \code{impact} project is to support two
critical aspects of library development.  The first is to make it very
easy for library developers to {\em publish} their work.  The second
is, at the same time, to make it easy for library consumers to both
find and install published libraries.

We also feel it is important  to reinforce best practices with respect
to model development.   For this reason, we have  made version control
an integral  part of  our solution.   Rather than  putting users  in a
position to have  to figure out how to make  \code{impact} work with a
version control  system, we've build \code{impact}  around the version
control system.   Not only  do users not  have to find  a way  to make
these technologies work together,  \code{impact} actually nudges those
not using  version control  toward solutions that  incorporate version
control.  In this way, we hope to demonstrate to people the advantages
of both \code{impact} and version control and establish both as ``best
practices'' for model development.

By creating a tool that makes it easy to both publish and install
libraries, we feel we are creating a critical piece of the foundation
necessary to establish a {\bf healthy ecosystem} for model
development.

\subsection{History}

Earlier, we mentioned that \code{impact} has been completely
rewritten.  In fact, the very first version of \code{impact} was just
a single Python script for indexing and installing Modelica
code\cite{impact-gist}.  It eventually involved into a multi-file
package that could be installed using the Python package management
tools.

\section{Requirements}

After building the original Python version, we gave some thought to
what worked and what didn't work well.  One issue we ran into almost
immediately was the complexity of installing the Python version of
\code{impact}.  Python is unusual in that it has two package managers,
\code{easy\_install} and \code{pip}.  It comes with
\code{easy\_install}, but \code{pip} is the more capable package
manager.  So in order for someone to install \code{impact}, they first
needed to install Python, then install \code{pip} and then install
\code{impact}.  This was far too complicated.  So we wanted to come up
with a way for people to install \code{impact} {\bf as a simple
  executable} without any runtime or prerequisites.

Another issue we ran into with the Python version was the fact that
there are two different and incompatible versions of Python being used
today.  Trying to support both was an unnecessarily inefficient use of
resources.  For this reason, we felt we needed to move away from
Python altogether.

We also had some difficulties in the Python version with support for
SSL under Windows \cite{python-ssl}.  Because we were doing lots of
``crawling'' (more on this shortly), we needed a platform that
provided {\bf solid HTTP client support}.

Although most users Modelica run their development tools and
simulations under Windows, there are several tools that support OSX
and Linux as well as Windows.  So as to not neglect users of those
tools and to support more cross-platform options, we also wanted to be
able to {\bf compile impact for all three major platforms}.

Furthermore, we wanted to provide a simple executable for all
platforms without having to have actual development machines for each
of these different platforms.  For this reason, {\bf cross
  compilation} between different platforms was an important
consideration was well.

Of course, we also wanted to have {\bf good performance}.  For most
package management related functions, the speed of the internet
connection is probably the biggest limitating factor.  So CPU
performance wasn't that high on the list.  But, as we shall discuss
shortly, the computational complexity of the dependency resolution
algorithm we implemented could lead to some computationally intensive
calculations for complex systems of dependencies.

For these reasons, we ultimately rewrote \code{impact} in Go.  Go is a
relatively new language from Google that stresses simplicity in
language semantics but, at the same time, provides a fairly complete
standard library.  You can think of Go as being quite similar to C
with support for extremely simple object-oriented functionality,
automatic garbage collection and language level support for CSP-based
concurrency.  With Go, we were able to satisfy all the requirements
above.

\section{Version Numbering}
\label{sec:numbers}

Before we dive into all the details associated with crawling,
indexing, resolving and installing, it is useful to take a moment to
briefly discuss versioning.  Modelica supports the notion of versions
through the use of the `version` and `uses` annotations.  These two
annotations allow libraries to explicitly state what version they are
and what versions of other libraries they use, respectively.

But there is one complication to the way Modelica deals with versions.
In Modelica, a version is simply a string.  This by itself isn't a
problem.  But it becomes a problem, as we will discuss in greater
detail shortly, when you need to understand relationships between
version.  In particular, there are two important things we would like
to determine when dealing with version numbers.  The first is an
unambiguous ordering of versions.  In other words, which, of any two
versions, is the ``latest'' version?  The second is whether a newer
version of a library is ``backwards compatible'' with a previous
version.  These are essential questions when trying to resolve
dependencies and the current string based approach to versions in
Modelica is not semantically rich enough to help us answer either of
these.

This issue is not unique to the Modelica world.  These same questions
have been asked for a very long time and various approaches have been
invented to deal with answering these questions.  One recent and
widely used approach is to employ what is called {\bf semantic
  versioning}\cite{semver}.  Semantic versioning is pretty much what
it sounds like, an approach to defining version numbers where the
version numbers have very explicit meanings associated with them.

A very simple summary of semantic versioning would be that {\bf all}
versions have exactly three numerical components, a major version
number, a minor version number and a patch.  A semantic version must
have all of these numbers and they must be .-separated.  For this
reason, the following versions are not legal semantic version numbers:
\code{1}, \code{1a}, \code{1.0}, \code{1.0-beta5}, \code{4.0.2.4096}.
Each of these number means something.  If you make a non-backward
compatible change, you must increment the major version.  If you make
a backward compatible version, you must increment the minor version.
If you make a change that should be completely compatible with the
previous version, you increment only the patch version.

There are additional provisions in semantic versioning to handle
pre-release versions as well as build annotations.  We will not
discuss those semantics here, but they are incorporated into our
implementation's treatment of version numbers.

Our use of semantic versioning is aligned with our goal of strongly
encouraging best practices.  It is important to point out that the use
of semantic versions is completely legal in Modelica.  In other words,
Modelica allows a wider range of interpretations of version numbers.
By using semantic versions, we narrow these interpretations but we
feel that this narrowing is much better for the developer since it
also provides meaning to the version numbers assigned to a library.

However, because Modelica libraries are free to use nearly any string
as a version number, we need to find a way to ``bridge the gap''
between past usage and the usage we are encouraging moving forward.
Although internally \code{impact} understands {\bf only} semantic
versions, it is still able to work with nearly all existing Modelica
libraries.  This is achieved through a process of ``normalizing''
existing versions.  When \code{impact} comes across versions that are
not legal semantic versions, it attempts to create an equivalent
semantic version representation.  For example, a library with a
version string of \code{1.0} would be represented by the semantic
version \code{1.0.0}.

For this normalization to work, it is important to make sure that the
normalization is performed {\em both} on the version number associated
with a library and on the version numbers of the libraries used.  In
other words, it must be applied consistently to both the
\code{version} and \code{uses} annotations.

\section{Indexing}

As mentioned previously, there are two main functions that
\code{impact} performs.  The first is making it easy for library
developers to publish their libraries and the other is making it easy
for consumers to find and install those same libraries.  Where these
two needs meet is the library index.  The index is {\bf built} by
collecting information about publish libraries.  The same index is
{\bf used} by consumers searching for information about available
libraries.

Building the index involves crawling through repositories and
extracting information about libraries that those repositories
contain.  In the following section we will discuss this crawling
process in detail and describe the information that is collected and
published in the resulting index.

\subsection{Sources}

Currently, \code{impact} only supports crawling GitHub\cite{github}
repositories.  It does this by usin the GitHub API\cite{gh-api} to
search through repositories associated with particular users and to
look for Modelica libraries stored in those repositories.  We will
shortly discuss exactly how it identifies Modelica libraries.  But
before we cover those details it is first necessary to understand
which {\em versions} of the repository it looks in.

Each change in a Git repository involves a {\em commit}.  That commit
affects the contents of one or more files in the repository.  During
development, there are frequent commits.  To identify specific
versions of the repository, a tag can be associated with that
version.  Each tag in the repository history that starts with a
\code{v} and is followed by a semantic version number is analyzed by
\code{impact}.

\subsection{Repository Structure}

For each version of a repository tagged with a semantic version
number, \code{impact} inspects to contents of that version of the
repository looking for Modelica libraries.  There are effectively two
ways that \code{impact} finds Modelica libraries in a repository.  The
first is to check for libraries in ``obvious'' places that conform to
some common conventions.  In addition, it also looks for a special
file in each repository to provide guidance on where to find Modelica
libraries in non-conventional cases.

\subsubsection{Conventions}

The following is a list of structural entities that \code{impact}
checks for when parsing repositories:

\begin{description}
  \item[./package.mo] The entire repository is treated as a Modelica
    package whose name is the name of the repository.

  \item[./<reponame>/package.mo] The directory \code{<reponame>}
    contains a Modelica package.

  \item[./<reponame> <ver>/package.mo] The directory \code{<reponame>}
    contains a Modelica package.

  \item[./<reponame>.mo] The file \code{./<reponame>.mo} is a file
    containing the library named \code{<reponame>}.

  \item[./<reponame> <ver>.mo] The file \code{./<reponame>.mo} is a file
    containing the library named \code{<reponame>}.
\end{description}

where \code{<reponame>} represents the name of the repository and
\code{<ver>} is the version number associated with the current tag.

\subsubsection{\code{impact.json}}
\label{sec:dirinfo}

For various reasons, library developers may not wish to conform to the
repository structure patterns discussed previous.  Furthermore, there
may be additional information they wish to include about their
libraries.  For this reason, a library developer can include an
\code{impact.json} file in the root of the repository directory that
provides additional information about the contents of the repository.
For example, a repository may contain two or more Modelica libraries
in subdirectories.  The \code{impact.json} file allows information
about the storage location of each library in the repository to be
provided by the library developer.  Furthermore, the author may wish
to include contact information beyond what can be extracted from
information about the repository and its owner.  These are just a few
use cases for why an \code{impact.json} file might be useful for
library developers.  A complete schema for the \code{impact.json} file
can be found later in Section \ref{sec:dirinfo_schema}.

\subsection{Handling Forks}

The Modelica specification implicitly assumes that each library is
uniquely identified by its name.  This name is used in both the
\code{version} and \code{uses} annotations as well as any references
in Modelica code (e.g., \code{Modelica} in \code{Modelica.SIunits}).
This assumption works well when discussing libraries currently loaded
into a given tool.  But when you expand the scope of your
``namespace'' to include all libraries available from multiple
sources, the chance for overlap becomes possible and must be dealt
with.

Previously, we mentioned the importance of supporting best practices
in model development and the specific need to accomodate version
control as part of that process.  Up until now, we have leveraged
version control to make the process of indexing and collection
libraries easier.  However, version control does introduce one
complexity as well.  That complexity is how to deal with {\em forks}.

A fork occurs when there are multiple perspectives on how development
should progress on a given project.  In some cases, rather than
reconciling these different perspectives, developers decide to proceed
in different directions.  When this happens, the project becomes
``forked'' and there are then (at least) two {\bf different} libraries
being developed in parallel.  Each of these libraries may share a
common name and perhaps even the same version numbers but still be
fundamentally different libraries.

A fork can arise for another, more positive, reason.  When someone
improves a library they may not have permission to simply fold their
improvement back into the original library.  On GitHub in particular,
it is extremely common for a library to be forked simply to enable a
third-party to make an improvement.  The author of the improvement
then sends what is called a {\em pull request} to the library author
asking them to incorporate the improvement.  In such a workflow, the
fork is simply a temporary measure to support concurrent development.
Once the pull request is accepted, the fork can be removed entirely.

Regardless of why the fork occurs, it is important that \code{impact}
accomodates cases where forking occurs.  This is because forking is a
very common occurrence in a healthy eco-system.  It indicates progress
and interest and we should not do anything to stiffle either of
these.The issue with forking is that the same name might be used by
multiple libraries.  In such cases, we need a better way to uniquely
identify libraries.

For this reason, \code{impact} records {\bf not only the library name,
  but also a URI associated with each library}.  In this way, the URI
serves as a completely unambiguous way of identifying different
libraries.  While two forks may have the same name, they will never
have the same URI.

\subsection{Schema}

We've mentioned the kinds of information \code{impact} collects while
indexing as well as the kind of information that might be provided by
library developers (via \code{impact.json} files).  In this section,
we will provide a complete description of information used by
\code{impact}.

\subsubsection{\code{impact\_index.json}}
\label{sec:index_schema}

The root of an \code{impact\_index.json} file contains only two
elements:

\begin{description}
  \item[\code{version}] A string indicating what version of impact
    generated the index.  The string is, of course, a semantic
    version.
  \item[\code{libraries}] The libraries field is an array.  Each
    element in the array describes a library that was found.  {\bf The
      order of the elements is significant}.  Libraries that occur
    earlier in the list take precedence over libraries that appear
    later.  This is important in cases where libraries have the same
    name.
\end{description}

For each library in the \code{libraries} array, the following
information may be present:

\begin{description}
  \item[\code{name}] The name of the library (as used in Modelica)
  \item[\code{description}] A textual description of the library
  \item[\code{stars}] A way of ``rating'' libraries.  In the case of
    GitHub, this the number of times the repository has been starred.
    But for other types of sources, other metrics can be used.
  \item[\code{uri}] A URI to uniquely identify the given library (when
    it shares a common \code{name} with another library)
  \item[\code{owner\_uri}] A URI to uniquely identify the owner of the
    library
  \item[\code{email}] The email address of the owner/maintainer of the
    library
  \item[\code{homepage}] The URL for the library's homepage
  \item[\code{repository}] The URI for the library's source code
    repository
  \item[\code{format}] The format of the library's source code
    repository (e.g., Git, Mercurial, Subversion, {\it etc.})
  \item[\code{versions}] This is an object that maps a semantic
    version (in the form of a string) to details associated with that
    specific version)
\end{description}

The details associated with each version are as follows:

\begin{description}
  \item[\code{version}] A string representation of the semantic
    version ({\it i.e.,} one that is identical to the key).
  \item[\code{tarball\_url}] A URL that points to an archive of the
    repository in \code{tar} format.
  \item[\code{zipball\_url}] A URL that points to an archive of the
    repository in \code{zip} format.
  \item[\code{path}] The location of the library within the
    repository.
  \item[\code{isfile}] Whether the Modelica library is stored as a
    file (\code{true}) or as a directory (\code{false})
  \item[\code{sha}] This is a hash associated with this particular
    version.  This is currently recorded by \code{impact} during
    indexing but not used.  Such a hash could be useful for caching
    repository information locally.
  \item[\code{dependencies}] This is an array listing the dependencies
    that this particular version has on other Modelica libraries.
    Each element in this array is an object with one field,
    \code{name}, giving the name of the required library and another
    field, \code{version}, which is the {\bf semantic version} of that
    library represented as a string (see previous discussion on
    normalization in \ref{sec:numbers}).
\end{description}

\subsubsection{\code{impact.json}}
\label{sec:dirinfo_schema}

As mentioned previously in Section \ref{sec:dirinfo}, each directory
can include a file named \code{impact.json} that provides explicit
information about Modelica libraries contained in that repository.
The root of the \code{impact.json} file contains the following
information:

\begin{description}
  \item[\code{owner\_uri}] A link to information about the libraries
    owner
  \item[\code{email}] The email address of the owner or maintainer
  \item[\code{alias}] An object that whose keys are the names of
    libraries and whose associated values are the unique URIs of those
    libraries.  This information can, therefore, be used to
    disambiguate between dependencies where there may be multiple
    libraries with that name.
  \item[\code{libraries}] This is an array where each element contains
    information about a library present in the repository.
\end{description}

For each library listed in the \code{libraries} field, the following
information may be provided:

\begin{description}
  \item[\code{name}] The name of the library
  \item[\code{path}] The path to the library
  \item[\code{isfile}] Whether the entity pointed to by \code{path} is
    a Modelica library stored as a file (\code{true}) or as a
    directory (\code{false}).
  \item[\code{issues\_url}] A link pointing to the issue tracker for
    this library
  \item[\code{dependencies}] An explicit list of dependencies for this
    library (if not provided, the list will be based on the
    \code{uses} annotations found in the package definition).
\end{description}

For each dependency, the following information should be provided:

\begin{description}
  \item[\code{name}] Name of the required library
  \item[\code{uri}] Unique URI of the required library
  \item[\code{version}] {\bf Semantic version} number of the required
    library (represented as a string)
\end{description}

\section{Installation}

The previous section focused on how \code{impact} collects information
about available libraries.  The main application for this information
is support installation of those libraries.  In this section, we'll
discuss the installation side of using \code{impact}.

\subsection{Dependency Resolution}

To understand the abstract problem behind the concept of a dependency,
we refer to the formal study undertaken in \cite{boender2011formal}.
There, a repository is defined as a triple $(R,D,C)$ of a set of
packages $R$, a dependency function $D : R \rightarrow
\mathcal{P}(\mathcal{P}(R))$, and a {\em conflict relation} $C
\subseteq R \times R$.  At that level, version numbers have been
abstracted to (distinguishable) packages: Every version yields a
distinctive package $p \in P$.

The dependency function $D$ maps a package $p$ to sets of sets of
packages $d \in D(p)$, where each set represents a way to provide one
required required feature of $p$.  In other words: If for each $d \in
D(p)$ at least one package in $d$ is installed, it is possible to use
$p$.

Currently, there is no way to express {\em conflicts} directly in a
Modelica package.  However, due to the existance of external libraries
(which could conflict in arbitrary ways), it is likely that such a
need will arise in the future.  Additionally, current Modelica makes
it impossible to refer to two different versions of a library from the
same model.  Hence, we consider different versions of the same package
conflicting ({\code impact} can will only install one version of each
package in a given context).

The dependency resolution of {\code impact} fits into Boender's model.
Therefore, the conclusions drawn in \cite{boender2011formal} can be
applied to {\code impact} as well:

The set of packages {\impact} installs for a given project needs to
fulfill two properties, Boender calls {\em abundance} and {\em peace}.
Informally, abundance captures the requirement that all dependencies
be met while peace avoids packages that are in conflict with each
other.  A set of packages that is peaceful and abundant is called {\em
  healthy} and a package $p$ is called {\em installable} w.r.t. a
given repository if and only if there exists a healthy set $I$ in said
repository such that $p \in I$.

The problem of finding such an installable set is however a hard one.
In fact, Boender proves by a simple isomorphism between the boolean
satisifieability problem and the dependency resolution that finding
such a set is NP-hard.  Fortunately, for the current typical problem
size, this does hardly matter.

The indexing process collects quite a bit of information about
avialable libraries.  Most of the complexity in implementing the
installation functionality in \code{impact} is in figuring out {\be
  what} to install.

For example, a user may wish to install the
\code{MotorcycleLib} library.  If the user doesn't specify a
specific version number, we assume they want the latest version which,
at the time of this paper, was \code{1.0.0}.  But it turns out the
version \code{1.0.0} depends on \code{Modelica 3.2.1} and
\code{Modelica\_Synchronous 0.92.0'}.

* Basic algorithm (show some graphs)

* Extracting explicit dependency information from Modelica

* Implicit dependency information from semantic versions

* Backtracking and why it isn't so bad (give some performance numbers here)

\subsection{Directory Structure}

* No version numbers

* In a version control context

\subsection{impact.proj}

* --save option?

* .gitignore and repopulation

* disambiguation

\section{Go Implementation}

* package structure

* 3rd party libraries
  - GitHub API
  - Semver

* GitHub API tokens

* CI

\section{Additional Features}

\subsection{Searching}

* Ordering (matching URI/source then by stars)  

\section{Future Development}

As already mentioned, there is currently no way to express conflicts between different packages. 
However, it is highly likely that such conflicting pairs will exist as more and more packages are published. 
For instance, two Modelica models might depend on different, specific versions of an external library that 
cannot be linked or loaded at the same time, an already published package might contain known bugs etc. 
Hence, {\code impact} could be extended by the means to express conflicts as well.

Boender introduces the notions of {\em strong dependencies} and {\em strong conflicts} to optimize the handling
of very large repositories. 
This kind of optimizations might not be necessary in the Modelica ecosystem right now, but could provide helpful 
performance enhancements in future versions of {\code impact}.

* Bitbucket

* Subversion

* Intranet applications

* Web based search

\section{Conclusion}

* Different files (impact.json, impact.proj, impactrc and their purposes)

%%% Remove the following line once we got real citations in place.
% \nocite{impact}

%--------------------------------------------------------------------------------
% References using bibtex
\small
\bibliographystyle{plainnat}
\bibliography{impact}
\normalsize

\end{document}
